\documentclass{article}[10pt]
\usepackage[utf8]{inputenc}%characters and symbols from different languages
\usepackage{graphicx} % Required for inserting images
\usepackage[margin=1in]{geometry}
\usepackage{amsfonts, amssymb, amsmath}
\usepackage{tikz,pgfplots,hyperref}
\usepackage{float}
\usepackage{multirow}
\usepackage{tabularx}
\usepackage{hyperref}
\usepackage{biblatex}% references package
\title{reasearch on GPS module}
\author{Zuzanna Parnicka}
\date{September 2023}


\begin{document}
\renewcommand{\labelenumii}{\arabic{enumi}.\arabic{enumii}}
\renewcommand{\labelenumiii}{\arabic{enumi}.\arabic{enumii}.\arabic{enumiii}}
\renewcommand{\labelenumiv}{\arabic{enumi}.\arabic{enumii}.\arabic{enumiii}.\arabic{enumiv}}

\maketitle
\section{What to consider before choosing GPS module}
\begin{center}
    \begin{tabularx}{0.8\textwidth}{
    | >{\centering\arraybackslash}X  
    | >{\centering\arraybackslash}X | }

    \hline
    \multicolumn{2}{|c|}{List of needed requirements} \\
    \hline
    Type of requirement& specific of the requirement for the project\\
    \hline
    size wristband & max 30 mm x 30 mm\\
    \hline 
    size boat & max 60mm X 60mm\\
    \hline
    update rate&1-10 Hz\\
    \hline
    power requirements&3.3/5V - we make a voltage regulator\\
    \hline
    number of channels&?\\
    \hline
    time to first start&?\\
    \hline
    antenna&best if included\\
    \hline
    accuracy&the more precise the better, budget limitation , at least 3 m horizontal accuracy \\
    \hline
    microcontroller compatible with&Arduino\\
    \hline
\end{tabularx}
\end{center}

\section{Considered options and their specifications}
\begin{enumerate}
    \item Spark Fun RTK-SMA\\
    The SparkFun RTK-SMA GPS module is very precise, up to 0.01m in horizontal accuracy. However,
    the price is 250 \$ which is out of the semster project budget. The module is advancted and can do RTK. 
    However, it is not needed for the prototype of the project.
    link to the datasheet: \url{https://cdn.sparkfun.com/assets/f/8/d/6/d/ZED-F9P-02B_DataSheet_UBX-21023276.pdf?_gl=1*150sgcj*_ga*MTAxMTI1MDc4OS4xNjgxNzE5OTE5*_ga_T369JS7J9N*MTY5Njg3NjM0Mi40LjEuMTY5Njg3NjU1OS42MC4wLjA}
       
    \item NEO - 6\\
    Due to its low price relative to the functionalities it offers and compatibility with arduino, 
    this module is indisputable choice for those who want to learn how GPS works.This module is based on NEO-6M chip from U-blox. 
    It has a Pover Save Mode that makes it suitable for a wristband locator. It is also the smallest chip amnog the others listed in a  \autoref{tab:Gps modules}
    It includes antenna with sensitivity patch of 161dBm.
    \item BN-220\\
    The needed requirements are fullfilled despite the price. Which for 1000 krones budget is too high if the shipping price is added.
link to datasheet: \url{https://files.banggood.com/2016/11/BN-220%20GPS+Antenna%20datasheet.pdf}

\end{enumerate}

\section{Comparison}
\begin{center}
    \begin{table}[h!]
    \begin{tabularx}{0.8\textwidth}{
    | >{\centering\arraybackslash}X 
    | >{\centering\arraybackslash}X 
    | >{\centering\arraybackslash}X  
    | >{\centering\arraybackslash}X | }

    \hline
    requirement&Spark Fun RTK-SMA&NEO - 6&BN-220 \\
    \hline
    size&X&X&X\\
    \hline
    update rate&X&&X\\
    \hline
    power&X&&X\\
    \hline
    desired number of channels&&&72\\
    \hline
    time to first start (cold/warm)&-148dBm/-157dBm&&26s/25s\\
    \hline
    antenna&&&\\
    \hline
    accuracy of min 3 m&X&X&X\\
    \hline
    compatible with Arduino&X&X&X\\
    \hline
    budget&not in the budget&&about 150 krones but expensive shipping\\
    \hline
\end{tabularx}
 \caption{GPS modules comparison }
        \label{tab:Gps modules}
\end{table}
\end{center}



\end{document}

