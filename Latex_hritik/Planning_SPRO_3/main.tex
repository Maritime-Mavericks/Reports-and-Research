\documentclass{article}
\usepackage{graphicx} % Required for inserting images
\usepackage{float}
\usepackage[margin=0.75in, bottom=1in, paperwidth=8.5in, paperheight=11in]{geometry}
\parindent 0pt

\title{Preliminary Ideas}
\author{Hritik Roy Chowdhury}
\date{\today}

\begin{document}

\maketitle
\vspace{1cm}

\section{\begin{Large} Aerial Projects \end{Large}}

\subsection{Rocket Landing}
\subsubsection*{Description}
In this idea, the project's goal will be to design an autonomous rocket which after reaching a maximum height (apogee), tries to land back to its launcher pad autonomously. This is inspired from the SpaceX Falcon 9's rocket boosters which landed autonomously. We could extended the project so that the rocket separates its booster from its payload and then land only the booster autonomously. This is what the SpaceX Falcon 9 did. 
\subsubsection*{Research Areas}
\begin{itemize}
    \item Fluid Dynamics: Rocket movement and calculations
    \item Feedback Systems: Feeding back data to sensors and actuators
    \item Control Engineering: Controlling rocket?
    \item Neural Networks: (some sort of Machine Learning) Another approach to landing
    \item Pneumatics: Opening, closing valves
    \item Accelerometers and Altimeters: Reading live-data in high frequency
    \item Numerical Modelling: (e.g Euler's Method, Monte-Carlo Simulations) For modelling rocket parameters
\end{itemize}
\subsubsection*{Pros and Cons}
\begin{center}
    \begin{tabular}{c|c}
         \textbf{\textit{Pros}} & \textbf{\textit{Cons}} \\
         Several New Skills &  Unfamiliar Areas \\
         Many Research Areas & Risk of Failure \\
         Great Opportunity & Permission Required? 
    \end{tabular}
\end{center}
\vspace{1cm}

\subsection{Steady Drone Projects}
\subsubsection*{Description}
Here, the goal is to build a drone which can sustain it's acceleration, velocity and position continuously even with high wind speeds. Furthermore, to create the autonomous criterion, the drone could be a:
\begin{enumerate}
    \item Delivery drone: for delivering goods
    \item Tele-presence: similar to a walking robot, but more agile
\end{enumerate}
\subsubsection*{Research Areas}
\begin{itemize}
    \item Drone Systems: how drones work
    \item Weight Optimization: carry a lot of load and make the drone weigh less
    \item Data-transfer: Tele-presence control
    \item Feedback Systems: Feeding back data to sensors and actuators
    \item Neural Networks: some sort of Machine Learning
    \item Accelerometers: measuring position, rotation etc in real time.
\end{itemize}
\subsubsection*{Pros and Cons}
\begin{center}
    \begin{tabular}{c|c}
         \textbf{\textit{Pros}} & \textbf{\textit{Cons}} \\
         New Skills &  Unfamiliar Areas \\
         Preparation for $4^{th}$ Semester & Permission Required? \\
          & 4th Semester already has Drones
    \end{tabular}
\end{center}




\section{\begin{Large} Marine Projects \end{Large}}

\section{\begin{Large} Land Projects \end{Large}}


\end{document}
