\documentclass{article}[10pt]
\usepackage[utf8]{inputenc}%characters and symbols from different languages
\usepackage{graphicx} % Required for inserting images
\usepackage[margin=1in]{geometry}
\usepackage{amsfonts, amssymb, amsmath}
\usepackage{tikz,pgfplots,hyperref}
\usepackage{float}
\usepackage{multirow}
\usepackage{tabularx}
\usepackage{hyperref}
\usepackage{biblatex}% references package
\title{Risk Assesment}
\author{Zuzanna Parnicka, Yiheng Liu}
\date{16 December 2023}


\begin{document}
\maketitle
This Risk Assesment is based on Standards applicable for medical devices. Which relates to the project that is report is about.
Below mentioned standards are taken into account.
\section{Standards to be satisfied}
\begin{enumerate}
    \item ISO 14971 (ISO 14971:2012)
    The development team may self access and discard any negligible risks for the purpose of the risk assesment.
    \item Directive 93/42/EEC
    This standard is above ISO 14971 in order of application.\\
    All risks must be minimized, regardless of size, and balance against benefit of the divice. Only nonacceptable risks need to be accounted in ISO 14971.
    \item Risk Reduction
    \begin{enumerate}
        \item I14 - risks need to be reduced "As Low as Reasonably Practicable", with economic considerations permitted (ALARP).
        \item D93 - risks reduced "As Far as Possible", and no economic considerations permitted.
        \item I14-6.5 - If residual risks are not acceptable and further risk control not practicable, assess: Does benefit outweigh risks.
    \end{enumerate}
\end{enumerate}
\section{Risk Management}
\begin{enumerate}
    \item For the prupose of this project, the only applicable part of the life cycle where any risks are great enuogh to warrant thw application 
    of this plan will be during the "intended use time". this time is defined as from the begging to the end of te demonstartion of the project, 
    whether digitally recorded or live. \\
    The process: the device is meant to be a life saving boat that helps the lifeguard to take the drowning/injured person in the water back to the shore.
    The boat is called by a lifeguard's bracelet, comes to the lifeguard position and collects them and the injured back to the shore. 
    \item Responsibilities and authorities of the project.
    All group memebers are meant to work on each part of the project at least in very low level. Each person is resposible for explaining
    their done tasks to other group members. Each part of the project has a supervisor, that is resposible for giving tasks and checking:
    \begin{enumerate}
        \item Management: Vojtech Ilcik, Zuzanna Parnicka
        \item Electronics: Al Muthanna Almoslem
        \item Programming: Hritik Roy Chowdhury
        \item Mechnical: Yieng Liu
    \end{enumerate}
    \item Reviewing tasks
    Each risk is assesed on rolling basis as new and possible hazardous situations are indentified throughout the project design and testing phases.
    The risk's acceptability will be assesed using the criteria matrix from Appendix (Figure X).
    \begin{enumerate}
        \item risk 1
        \item risk 2
    \end{enumerate}
    \item Risk Acceptability, including risks with unpredictable probability of occurence and those of which harm cannot be estimated.
    The criteria is based on the matrix (Figure X) in Appendix.
    \begin{enumerate}
        \item risk 1
        \item risk 2
    \end{enumerate}
    \item Verification activities.
    Verification of the above data is done via data from manufacturer, data from typical operations of similar purpose and calibre, as well as data generated while testing the device itself.
    Figure X in Appendix.
    \item Review of production and postproduction information.
    Because the product has a very short life cycle, this section is not applicable. However, data retained from any testng during the design and
    prototyping phase will be reviewed and used for risks.



\end{enumerate}    


\end{document}